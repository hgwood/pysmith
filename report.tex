\documentclass[10pt,twocolumn]{article}

\usepackage{graphicx}
\usepackage[english]{babel}
\usepackage[utf8]{inputenc}
\usepackage[cm]{fullpage}

\title{PySmith: A Random Python Code Generator}
\author{Hugo Wood}

\begin{document}

\maketitle

\section*{Where we left off}

The first version of PySmith, developed during the Fall quarter, had one main 
drawback: it borrowed the principles of CSmith and blindly applied them to 
Python. This process resulted in the generator being severely limited by its 
need to track types. While this is completly fine (and necessary) when 
generating code in statically-typed languages, it is crippling when producing 
code in dynamically-typed languages such as Python. It applies large additional 
constraints to the generated code, decreasing its variety, and disables a whole 
chunk of language features based on dynamic typing. An example of such a feature
is the possibility of declaring a variable inside a branch or a loop, and have 
this variable reachable outside the scope of that branch or loop.

\section*{Embracing Dynamic Typing}

The solution to this problem is to stop keeping track of types. However,
this gives the generator too much freedom. The generated code has calls to 
non-existing members, uses operators with invalid types, calls functions with 
completly random parameter etc. Notice that this type of code has no undefined
behavior and will just raise exceptions, consequently it can still be use to 
test interpreters. Yet, generating 200 lines code so that the first one raises
an exception is not very interesting.

So how can the generator know if it can output \verb|a+b| without knowing what 
are the types of \verb|a| and \verb|b|? Well, there is only one way. The code 
has to be run. If an error is thrown, then the statement is not valid. This 
is a native feature of Python, exposed as the \verb|exec| statement. This 
statement has to be supplied with code (as a string) and two mappings from 
strings to objects representing the global and local variables respectively.
The code is interpreted within the context of these mappings, which get updated
accordingly. If the code raise an error, it bubbles up to the caller.

This solution was implemented. Every statement that gets generated is run in a 
context (separate from the one the generator itself is running in). Depending on
a user setting, an erroneous statement can be dropped or kept. Obviously, the 
generation process is a lot slower, but it is also much more "Pythonic" and 
can reach more into the language. Also, the code is signigicantly simpler and 
smaller.

The problem now is that a majority of statements gets dropped. In order to solve
this, a number of "guards" were implemented. For example, when generating a 
division, instead of picking two variables at random, the program 
queries the current context for two variables that expose the methods \verb|__div__| and 
\verb|__rdiv__| respectively \footnote{I have learnt since that this is not 
exactly the way to find two compatible operands, but this technique works quite well.}, 
and cheks that the second one is not 0. Such "guards" prevents the generation 
of erroneous statements ahead of their execution, hence reducing those from nearly 
every statement to just a few.

\section*{Refactoring}

The project went through extensive refactoring. An emphasis was put on 
simplicity. The model classes of language constructs
are now delegated to \verb|ast| module, part of Python's standard library. All 
user-definable rules are stored in the \verb|rules| module.

\section*{New Constructs and Features}

\begin{itemize}
	\item{Augmented assignments.}
	\item{Variadic functions. Both classic and keyword extra arguments are 
		supported.}
	\item{While loops. Termination is guaranteed through an additional exit 
		condition of the form \verb|x < A| where $A$ is integer constant and $x$ is 
		always incremented in the last statement of the loop. Moreover, the 
		\verb|continue| keyword is not allowed.}
	\item{Classes. Single and multiple inheritance are supported. Old-style and
		new-style classes are supported.}
	\item{String, tuple, list and dict literals.}
	\item{Built-ins read and write. All built-in types and functions can be used
		and redefined in generated programs.}
\end{itemize}

\section*{Results}

Unfortunately (fortunately?), no bugs were found in the 4 major Python
interpreters.

\end{document}